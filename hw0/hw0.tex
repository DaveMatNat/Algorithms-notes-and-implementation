%% HOW TO USE THIS TEMPLATE:
%%
%% Type your solution to each problem part within
%% the \begin{solution} \end{solution} environment immediately
%% following it.  Use any of the macros or notation from the
%% header.tex that you need, or use your own (but try to stay
%% consistent with the notation used in the problem).
%%
%% If you have problems compiling this file, you may lack the
%% Header.tex file (available on the course box page), or your system
%% may lack some LaTeX packages.  The "exam" package (required) is
%% available at:
%%
%% http://mirror.ctan.org/macros/latex/contrib/exam/exam.cls
%%
%% Other packages can be found at ctan.org, or you may just comment
%% them out (only the exam and ams* packages are absolutely required).


% The "answers" option causes the solutions to be printed.
% The pdf for this homework that's on blackboard was compiled without the
% answers flag for compactness. You need to include it.
%\documentclass[11pt,addpoints]{exam}
\documentclass[11pt,addpoints,answers]{exam}

% required macros -- get latest hwheader.tex file from blackboard if compiling
\input{hwheader}

\hwheader{Homework 0}
{David Nathanson}
{Spring 2023, Bhakta}
{CS 315, Algorithms}
{Due: 11:59pm Wednesday, Jan 18}

% VARIABLES

\begin{document}

\pagestyle{head}                % put header on every page

\noindent 
This assignment is a refresher on the basics for logic and proofs.
It is basically a few of my old CS 222 homeworks cobbled together.
You do not have to submit solutions to this assignment at any point,
nor will I grade them if you do.

Though this assignment won't be graded, you are welcome and encouraged to see me
in office hours to discuss these questions.

\begin{questions}
  \setcounter{question}{-1}

\question (EXAMPLE) Prove or disprove the following statements: 
\begin{parts}
  \part (EXAMPLE) $\star$ For all integers $x,y$, if $x$ and $y$ are both
  even, then $x+y$ is also even.

  This is a "good" clean proof of this fact.
  \begin{solution}
    This fact is true.

    \begin{proof}[good] 
      Let $x$ and $y$ be even integers.

      By the definition of even, there exist integers $k,j$ such that $x=2k, y=2j$.

      Therefore $x + y = 2k + 2j = 2(k+j)$.

      Since $k$ and $j$ are integers, $k+j$ is an integer, so $x+y=2(k+j)$ is
      twice an integer and therefore satisfies the definition of even.
    \end{proof}
  \end{solution}

  (Notes regarding the above proof)
  \begin{itemize}
    \item In the first step, we let x and y be \emph{arbitrary} even integers.
      This means that we can conclude that this result applies to \emph{all}
      pairs of even integers.
    \item In order to talk about things being even, we need to use the
      definition of even - an integer $x$ is even iff $x = 2y$ for some
      integer $y$.
    \item We use the definition two ways. We *use* the definition of even to
      establish facts about $x$ and $y$ at the beginning of the proof. We also
      show that $x+y$ *satisfies* the definition of even at the end of the
      proof.
  \end{itemize}

  This is a bad, incorrect "proof" of the same fact.

  \begin{solution}
    \begin{proof}[bad]
    $x=2k$ for some $k$, and $y=2k$ for some $k$ by definition of even; 
    $x+y=4k$ by definition of even.
    \end{proof}
  \end{solution}

  (Notes regarding this proof attempt).
  \begin{itemize}
    \item Hard for me to tell if you decided to prove that this is true, or
      disprove it as false.
    \item Not enough words are used to make absolutely 100\% clear what the
      proof writer is doing and why.
    \item The same variable name $k$ is used for two (potentially) different
      integers, resulting in a (possibly false) conclusion that $x+y$ is a
      multiple of 4.
  \end{itemize}

  This is a blatantly wrong ``proof'' of the same fact.

  \begin{solution}
    This fact is true.
    \begin{proof}[really bad]
      Let $x,y$ be even integers.
      Assume $x=2$ and $y=2$. Then $x+y=4$, which is even.
    \end{proof}
  \end{solution}

  (Notes regarding this proof attempt).
  \begin{itemize}
    \item The statement that we are trying to prove is about all even numbers.
      The assumptions made about $x$ and $y$ make this no longer a general
      statement about all even integers.
  \end{itemize}


  \part (EXAMPLE) $\star$ For all integers $x$ and $y$, $x$ and $y$ are both
  even if and only if $x+y$ is even.

  This is an incorrect solution.

  \begin{solution}
    This statement is true.
    Say $x + y = 4$. Then assume $x = 2, y = 2$. $x$ and $y$ are both even.
  \end{solution}
  (Notes on above incorrect solution)
  \begin{itemize}
    \item Again, we're making a statement about general integers. This means
      that this not a valid proof, because it only asserts something for a
      specific case of $x$ and $y$.
  \end{itemize}


  This is a correct solution.

  \begin{solution}
    This statement is false.
    As a counterexample, consider the case where $x=1,y=1$. 
    Then we have that $x+y=2$, which is even, although $x$ and $y$ are not
    even.
  \end{solution}

  (Notes on the correct solution)
  \begin{itemize}
    \item Here we disproved a statement by finding a counterexample. 
    \item The statement was of the form $\forall x,y p(x,y) \iff q(x,y)$. A
      counterexample would be a case of $x, y$ where $p(x,y) = True$ yet
      $q(x,y) = False$, which is what we found.
  \end{itemize}


  \part (EXAMPLE) $\star$ There is no smallest integer.

  Bad solution
  \begin{solution}
  \begin{proof}
    This statement is true.
    The set of integers goes to $-\infty$, so there's no smallest one.
  \end{proof}
  \end{solution}
  (Notes on above solution)
  \begin{itemize}
    \item This may be good intuition, but it's a bad proof.
    \item Parts of this are vague, what do you mean a set ``goes to $-\infty$?
      A set isn't necessarily a sequence of integers in order.
    \item Asserting that the set of integers ''goes to $-\infty$``, even if
      that statement was made rigorous, would effectively only be a rephrasing
      of our statement of ''There is no smallest integer``. It is not
      justifing why this statement is true based on simpler principles, it is
      only rephrasing the statement as a different, equivalent statement that
      is provided without justification and might also need proof.
  \end{itemize}

  Good solution
  \begin{solution}
  \begin{proof}
    This statement is true.
    Proof by contradiction. 
    
    Assume that there is a smallest integer. Call it $x$.

    If $x$ is an integer, then $x-1$ is an integer. However, $x -1 < x$.

    This contradicts the fact that $x$ is the smallest integer.
  \end{proof}
  \end{solution}

  (Notes on above solution)
  \begin{itemize}
    \item We announce that we are doing a proof by contradiction, to make it
      clear to the reader what we are doing and why.
    \item The opposite of the goal is assumed in the first step.
    \item Once we know something with a given property exists, we can give it
      a name:  $x$.
    \item We can do math using this hypothetical $x$ to arrive at a
      contradiction, namely that $x-1$ would be a smaller integer.
    \item The facts that we are using ``x is an integer -> x-1 is an
      integer'', and $x-1 < x$ are well known facts of integer arithmetic.
      They are more fundamental than ``there is no smallest integer''. These
      kinds of statements could be explained further all the way down to the
      axioms of set theory, but this is generally unnecessary. The depth to
      which you need to explain facts depends on your audience (math vs CS,
      undergrads vs researchers, etc.).
  \end{itemize}
\end{parts}

\question[30]
Prove or disprove the following statements. The proofs should only need to be
a few sentences long. Some of these may be somewhat tricky. You may assume the
following common high school level facts about arithmetic without further
proof.

\begin{itemize}
  \item Definition of even, odd, rational, irrational.
  \item The sum and product of integers is an integer.
  \item Addition and multiplication is commutative and associative.
  \item The fact that many numbers like $\sqrt{2}, \pi, \ldots$ are irrational.
\end{itemize}

\begin{parts}
  \part If $x$ is a rational real number, then $2x + 1$ is also rational.
  \begin{solution}
    
  \end{solution}

  \part If $x$ is an irrational real number, then $2x + 1$ is also irrational.
  \begin{solution}
    
  \end{solution}

  \part For all irrational numbers x,y, $x\cdot y$ must be irrational.
  \begin{solution}
    
  \end{solution}


  \part For all real $y$, there exists a non-negative real $x$ such that $x^2
  = y$.
  \begin{solution}
    
  \end{solution}

  \part For all non-negative integers $x,y$ such that $x < y$, there exists an
  integer $z$ such that $x^2 < z < y^2$.
  \begin{solution}
    
  \end{solution}

  \part Let $a_1, a_2, \ldots a_n$ be a set of n real numbers. The mean (or
  average) of these is defined to be $\bar{a} = \frac{\sum_{i=1}^n a_i}{n}$.
  Claim: For any such list of real numbers $\set{a_i}$,
  there must be some $i, 1 \leq
  i \leq n$ where $a_i \geq \bar{a}$.
  \begin{solution}
    
  \end{solution}

\end{parts}


\question[30]
  You are Detective Holmes, and you are investigating the stabbing murder of
  Garrett Lambert. There were 7 total people who entered the study building,
  Abed, Britta, Chang, Dean, Elroy, Frankie, and Garrett, the
  victim.  The six people other than Garrett are suspects.

  The building where the murder happened has three study rooms. Each of the 7
  people above went into a study room.
  The murderer was alone with Garrett in one of these three rooms.
  Unfortunately, we don't know which room Garrett went to or with who.

  One of these six suspects is the murderer. All of the suspects have given
  statements below. The innocent people give completely true statements, but
  the murderer may or may not be lying.
  \begin{itemize}
    \item Abed: ``Elroy and Frankie weren't together''
    \item Britta: ``I was with Dean''
    \item Chang: ``I was not with Britta or with Elroy''.
    \item Dean: ``Abed and Chang were in the same room''.
    \item Elroy: ``I wasn't with Britta''
    \item Frankie: ``I wasn't with Dean, but Elroy was.''
  \end{itemize}

  Deduce who the murderer is, and PROVE that the murderer could only be the
  person you think it is by writing a clear and concise statement to
  the police at Scotland Yard (FYI the police at Scotland Yard hate you).

  You should be mindful of the various proof techniques, use them
  appropriately. Rather than doing a massive proof by cases, try to reduce the
  number of cases you have to consider by making appropriate observations.
  \begin{solution}
    
  \end{solution}


\question (EXAMPLE)
For positive integer $n$, find a simple formula for $\sum_{i=1}^n
\frac{1}{2^i} = \frac{1}{2} + \frac{1}{4} + \frac{1}{8} + \ldots +
\frac{1}{2^n}$, and prove it.
  \begin{solution}
    Claim: $\sum_{i=1}^n \frac{1}{2^i} = 1 - \frac{1}{2^n}$.

    We prove this by induction.

    Basis Step: for $n=1$, $\frac{1}{2} = 1 - \frac{1}{2^1}$ as desired.

    Indctive Step:
    For a \emph{fixed} $n \geq 1$, assume that $\sum_{i=1}^n \frac{1}{2^i} = 1
    - \frac{1}{2^n}$.

    Then,
    \begin{align*}
      \sum_{i=1}^{n+1} \frac{1}{2^i}
      &= \frac{1}{2^{n+1}} + \sum_{i=1}^{n} \frac{1}{2^i} \tag{Expand the sum}\\
      &= \frac{1}{2^{n+1}} + 1 - \frac{1}{2^{n}} \tag{Inductive Hypothesis} \\
      &= 1 - (\frac{1}{2^{n}} - \frac{1}{2^{n+1}}) \tag{Algebra} \\
      &= 1 - \frac{1}{2^{n+1}} \tag{Simplification}
    \end{align*}

    Thus $\sum_{i=1}^{n+1} \frac{1}{2^i} = 1 - \frac{1}{2^{n+1}}$. This
    concludes the inductive step.

    Therefore, we conclude by induction that
    $\sum_{i=1}^n \frac{1}{2^i} = 1 - \frac{1}{2^n}$ for all positive $n$.
  \end{solution}

\question (EXAMPLE)
Show by induction that the number of distinct 2 element subsets of an $n$
element set is $\frac{n(n-1)}{2}$ for $n \geq 2$.
  \begin{solution}
    Basis Step: for $n=2$, there is only one 2-element subset of a 2-element
    set. Also, $\frac{2(2-1)}{2} = 1$ as desired.

    Indctive Step:
    Say that the claim is true for a fixed $n$. Consider a set $A$ with $n+1$
    elements. 
    Let those elements be referred to as $a_1, a_2, \ldots a_{n+1}$.
    We separate the 2-element subsets of $A$ into two groups.
    \begin{itemize}
      \item The set of 2-element subsets of $A$ that include $a_{n+1}$. 

        Knowing that $a_{n+1}$ is one element of the 2-element subset, the
        only possible subsets are $\set{a_1, a_{n+1}},
        \set{a_2, a_{n+1}}, \ldots \set{a_n, a_{n+1}}$.

        There are exactly $n$ such subsets.
      \item  The set of 2-element subsets of $A$ that do not include
        $a_{n+1}$. 
        
        Since we do not include $a_{n+1}$, these subsets are
      exactly the subsets of $\set{a_1, a_2, \ldots a_n}$, which has $n$
      elements. 
      
      By the inductive hypothesis, there are $\frac{n(n-1)}{2}$ of
      these sets.
    \end{itemize}
    Therefore, the total number of 2-element subsets of $A$ is
    $\frac{n(n-1)}{2} + n = \frac{n(n-1) + 2n}{2} = \frac{(n+1)(n)}{2}$ as
    desired.
  \end{solution}

  \question (EXAMPLE)
  Find the flaw in this inductive proof.

  Claim: Let $n$ be a finite integer. In every set of $n$ people, all the people
  in the set share the same height.
   
  Proof: We prove this by induction.

  Base Case: If $n = 1$, then it's true that the one person in a set of $1$
  element has the same height as everyone in the set.

  Inductive step:
  Say that, for fixed $n$, every set of $n$ people must have the same height.

  Consider a set of $n+1$ people. Number each person $a_1 \ldots a_{n+1}$.
  The subset $\set{a_1 \ldots a_n}$ has $n$ people, and therefore all of them
  must have the same height by the inductive hypothesis. The subset
  $\set{a_2 \ldots a_{n+1}}$ also has $n$ elements, and they must all have
  height as well by the inductive hypothesis. Since the two sets overlap,
  it follows that all $n+1$ people in the group must have the same height.

  Therefore, we conclude by induction that all finite sets of people have the
  same height.

  \begin{solution}
  \end{solution}

  \question (OPTIONAL)
  Clearly state the flaw in this inductive proof. (Pretend that someone
  you hate is trying to make the following argument. Convince them why
  they are wrong.)

  Claim: $2^n = 1$ for all $n \geq 0$.
   
  Proof: We prove this by strong induction.

  Base Case: If $n = 0$, then it's true that $2^0 = 1$.

  Inductive step:
  Assume that for fixed $n$ and all $i : 0 \leq i \leq n$ that $2^i = 1$.
  Then $2^{n+1} = \frac{ 2^n \cdot 2^n }{2^{n-1}}$. By the inductive
  hypothesis, $2^n$ and $2^{n-1}$ are both $1$. Therefore $2^{n+1} = \frac{1
  \cdot 1}{1} = 1$.

  Therefore, we conclude that $2^n = 1$ for all $n \geq 0$

  \begin{solution}
  \end{solution}


\question[30]
  A recurrence sequence $a_1, a_2, \ldots$ is defined as: $a_0 = 0$, $a_1 = 1$,
  and for all $i \geq 2$, $a_i = 2a_{i-1} - a_{i-2} + 2$. Find a simple closed
  form formula for $a_i$, and prove that your statement holds for all $n$ by
  induction.
  \begin{solution}

  \end{solution}

\question[30]
  For a graph $G = (V,E)$, let $deg(v)$ be the number of neighbors of vertex
  $v$. Prove by induction that $\sum_v deg(v) = 2|E|$ for all graphs $G$.

  Hint: Prove this by induction on the number of edges. In other words, start
  proving that this fact is true for graphs with $0$ edges as a base case,
  then $1$ edge, then $2$, etc. Remember to ``look backwards'' to previously
  proven statements, not ``forwards''.

  \begin{solution}

  \end{solution}

\end{questions}
\end{document}
